\chapter{Introducción}
\label{ch:introduccion}

\section{Contexto}

En la era digital, las empresas y organizaciones están enfrentando una creciente demanda de la sociedad de aplicaciones y servicios en línea que sean capaces de manejar grandes cantidades de tráfico y datos. Debido a esto, los desarrolladores y arquitectos están adoptando e investigando tecnologías que logren satisfacer estas necesidades de manera eficaz, rápida e incurriendo en los menores gastos posibles.

\textbf{El uso de contenedores}

El uso de los contenedores (o \textit{containerización}) hizo posible el despliegue de aplicaciones de manera reproducible y aislada y se erigió como alternativa a las máquinas virtuales. Los contenedores dotaban de todas las características que te ofrecían otras alternativas, sumado a la reproducibilidad y al menor uso de recursos en comparación, debido a que no necesitan un sistema operativo completo para funcionar.

\textbf{Microservicios y orquestación de contenedores}

Con tecnologías como Kubernetes, y aprovechando las características antes citadas de los contenedores, el uso de microservicios se ha convertido en una forma efectiva de diseñar y desplegar aplicaciones escalables y seguras. Los microservicios son componentes autónomos y atómicos que interactúan entre sí para proporcionar un servicio completo. La orquestación de contenedores, como Kubernetes, permite a los desarrolladores gestionar y coordinar el despliegue de estos microservicios de manera eficiente. La orquestación te otorga beneficios clave a la hora de construir aplicaciones y servicios:

\begin{itemize}
    \item Gestionar el despliegue y la eliminación de contenedores
    \item Monitorear el rendimiento de los microservicios
    \item Escalar los recursos según sea necesario
    \item Asegurarse de que los contenedores estén funcionando correctamente
\end{itemize}

\textbf{Serverless y \textit{Functions as a Service}}

En este contexto, Serverless (también conocido como FaaS, Functions as a Service) se está convirtiendo en una metodología arquitectónica innovadora que permite a los desarrolladores crear aplicaciones sin la necesidad de administrar servidores. En lugar de eso, los desarrolladores pueden enfocarse en escribir código y dejar que las plataformas de Serverless manejen el resto.

Las plataformas de Serverless como AWS Lambda, Google Cloud Functions, Azure Functions, o incluso plataformas \textit{on-premises} como OpenFunctions o Knative, permiten a los desarrolladores:

\begin{itemize}
    \item Crear funciones pequeñas y atómicas que se ejecutan solo cuando son necesarias
    \item Utilizar recursos en la nube según sea necesario, sin la necesidad de administrar servidores
    \item Escalar las funciones según sea necesario para manejar un gran volumen de tráfico
\end{itemize}

Las ventajas clave de utilizar Serverless incluyen: una significativa reducción del coste de \textit{hosting} (en el caso de estar usando un \textit{cloud} de pago como Amazon Web Services) o, en caso de ser \textit{on-premises} una reducción de los recursos usados; mayor escalabilidad al poder escalar las funciones según sea necesario con base en el volumen de tráfico que haya; mayor flexibilidad al delegar la decisión a los desarrolladores para elegir la plataforma y la tecnología de desarrollo; y una mejora en la seguridad ya que, como bondad heredada de los contenedores, las funciones son aisladas y seguras.


\textbf{WebAssembly en entornos distribuidos}

En este contexto, WebAssembly (WASM) se está convirtiendo en una tecnología innovadora que puede unificar y simplificar el proceso de desarrollo y despliegue de aplicaciones distribuidas. WASM es un formato de código compilado que permite a los desarrolladores crear aplicaciones web que sean capaces de ejecutarse en diferentes entornos y plataformas, sin la necesidad de traducción o interpretación. Pese a que en su origen WASM fue concebido para poder ejecutar código en clientes web distintos, la creación de WebAssembly System Interface (WASI) hizo posible la ejecución de código compilado a WASM en el entorno del servidor. WASI permite a estos programas hacer llamadas al sistema donde se ejecuta, como entrada y salida, o lectura y escritura de ficheros, entre otras.

\textbf{El problema subyacente}

En todos estos casos, existe un problema de base que es que pese a que los contenedores han resultado ser una herramienta más que fiable, que tiene fuertes mejorías frente a los servidores y máquinas virtualizadas, siguen teniendo espacio a mejora. Actualmente, las imágenes de Docker siguen siendo muy pesadas, los tiempos de arranque y de build son generalmente lentos y encontramos que muchos entornos siguen siendo muy inseguros por errores humanos en la configuración de cada contenedor.

 
\section{Objetivos}

El objetivo principal de este proyecto no es otro que probar que la implementación de aplicaciones WebAssembly en contenedores puede brindar mejores resultados en términos de rendimiento, escalabilidad y seguridad en comparación con otras tecnologías de contenedores. En este sentido, se quiere evaluar y comparar los resultados de implementar una aplicación utilizando WebAssembly en contenedores con otros contenedores que utilizan diferentes tecnologías y enfocarse en la evaluación de su desempeño, escalabilidad y seguridad.

Además de este, se han establecido diferentes objetivos derivados del principal para la realización de este proyecto, que aporten una profundidad extra a los resultados obtenidos, a la par que contribuir comunitariamente al buen desarrollo de los proyectos WASM de código abierto.

\subsection{Principal}

Evaluar la viabilidad y el rendimiento de WebAssembly usando WasmEdge en entornos de contenedores, mediante el desarrollo de pruebas comparativas y benchmarks que analicen tiempos de ejecución, tamaños de imagen, y la identificación de bugs y compensaciones (trade-offs) en el uso de contenedores. Además, se pretende hacer una comparativa con la facilidad de uso de contenedores basados en Wasm frente a contenedores con imágenes basadas en distribuciones Linux convencionales.

\subsection{Derivados}

Los siguientes objetivos son derivados del objetivo principal, algunos simplemente porque son necesarios para la realización de este, y otros como consecuencia de su realización.

\subsubsection{1. Fomento y contribución al desarrollo Open Source de WasmEdge}

Fomentar y fortalecer la contribución al ecosistema de WasmEdge mediante la implementación de mejoras clave en la documentación con base en el trabajo hecho para este proyecto. A su vez, se mantendrá una colaboración activa con la comunidad de desarrolladores que se lleva a cabo en el Discord oficial, donde se han debatido diferentes aspectos de este \textit{runtime}.

\subsubsection{2. Análisis arquitectural de Docker y Kubernetes}

Realizar un análisis exhaustivo de la arquitectura y evolución de Docker y Kubernetes, enfocándose en sus componentes fundamentales, diferencias en imágenes, y sus respectivos \textit{runtimes}, para proporcionar una guía comprensiva que facilite la adopción y optimización de estas tecnologías en entornos de desarrollo y producción. Además, este análisis es necesario para entender de manera profunda como se integran los entornos de ejecución de WebAssembly y como funcionan.

\subsubsection{3. Análisis del desarrollo de componentes en WASM y WASI}

Realizar una investigación exhaustiva sobre el desarrollo de aplicaciones y componentes en WebAssembly (WASM) y WebAssembly System Interface (WASI), incluyendo su estado actual, lenguajes compatibles, y su integración con contenedores. El resultado debe de ser una vista general del estado del desarrollo de estas tecnologías, su actual viabilidad de adopción en proyectos en producción, y sus trade-offs a la hora de adoptarlo.

\subsubsection{4. WebAssembly en plataformas de contenedores especificas}

Evaluar y optimizar el uso de WebAssembly con WasmEdge en Kubernetes y Docker utilizando \textit{runtimes} como containerd en diferentes plataformas (microk8s, minikube, k3s y Docker Desktop), mediante el desarrollo de pruebas comparativas y benchmarks que analicen la facilidad de uso, tiempos de ejecución, tamaños de imagen, y la identificación de bugs y compensaciones (trade-offs) en el uso de contenedores, complementado con la documentación del proceso de despliegue usando \textit{RunWasi}.

\subsubsection{5. WebAssembly en entornos serverless}

Evaluar la viabilidad y el rendimiento de WebAssembly usando WasmEdge en entornos Kubernetes con containerd y Knative, mediante el desarrollo de pruebas comparativas y benchmarks que analicen el tiempo de ejecución, tiempo de startup, tamaños de imagen, identificación de bugs y compensaciones (trade-offs), y la verificación de las expectativas de WebAssembly en arquitecturas serverless. Los entornos Serverless deberían de ser el escenario idílico para la ejecución de microservicios basados en WebAssembly debido a su naturaleza que reduce teóricamente el tiempo de arranque (cold-start) y de ejecución.

\section{Motivación}

La motivación detrás de este trabajo radica en la experiencia propia a la hora de desarrollar y desplegar microservicios en Kubernetes, con varias capas de extra como Istio. Tienes que andar manteniendo cada \textit{deployment} teniendo cuidado de las vulnerabilidades que pueda tener cada imagen creada para cada \textit{pod}, de los tiempos de creación, arranque y borrado de cada uno, y en general de los \textit{trade-offs} de tener muchos contenedores Linux \textit{hosteados} en un mismo cluster. Docker ahora mismo es la mayor aplicación de gestión de contenedores del mundo, y están apostando mucho por WebAssembly, porque confían en que pueden ser el futuro que venga a arreglar los problemas que tienen los contenedores actuales. 

El proyecto propuesto busca evaluar la viabilidad y el rendimiento de WASM utilizando WasmEdge en diferentes entornos de contenedores, con el fin de desarrollar soluciones más eficientes y escalables para el desarrollo de aplicaciones. Los resultados esperados son una mayor comprensión del potencial de WASM en entornos de contenedores, así como la identificación de oportunidades de mejora y optimización para su implementación efectiva.

Muchos de los principales creadores de los runtimes de WASM que existen en la actualidad afirman que, pese a estar esta tecnología en una fase muy temprana de desarrollo, ya se encuentran claras mejorías frente a los contenedores Linux. Previamente y durante la realización de este proyecto se han encontrado diferentes artículos y publicaciones que eran contradictorios en los resultados y conclusiones de cada uno. Además, en este trabajo se quiere ir un poco más allá e intentar tener un contacto directo con los desarrolladores detrás de WasmEdge, para que puedan ir dando feedback o ayudándome a entender el razonamiento y motivos detrás de los resultados extraídos, y cuáles son sus planes de cara a futuro.

En este sentido, esta investigación se enfoca en la aplicación práctica de los conceptos teóricos aprendidos durante el desarrollo del título de grado, y contribuye a la formación de competencias en áreas como el desarrollo de aplicaciones, la ingeniería de software y la innovación tecnológica.

\section{Justificación}

En los últimos meses, he estado involucrado en varios proyectos que han sido desarrollados para entornos distribuidos y en microservicios. Dada esta experiencia, he podido observar que aún hay un gran espacio de mejora en este tipo de arquitecturas. Si bien, cuando se aplican adecuadamente frente a los monolitos, pueden resultar muy positivas para la experiencia del usuario final, en la mayoría de las ocasiones pueden representar una complicación adicional para el desarrollador detrás de estos.

WebAssembly (Wasm) ha emergido como una tecnología prometedora que puede abordar algunos de los desafíos inherentes a los entornos distribuidos. Originalmente diseñado para ejecutar código de manera eficiente en navegadores web, WebAssembly ha evolucionado para soportar una amplia gama de aplicaciones más allá del navegador, incluidas las aplicaciones en servidores y dispositivos IoT. Su capacidad para ejecutar código casi nativo con alta eficiencia y seguridad lo convierte en una opción atractiva para arquitecturas de microservicios.

La principal motivación detrás de este estudio es explorar cómo WebAssembly puede mejorar el desarrollo y la implementación de sistemas distribuidos. Las ventajas potenciales incluyen una mayor portabilidad del código, una menor huella de memoria y un rendimiento mejorado en comparación con las soluciones tradicionales. Además, WebAssembly promete una mayor seguridad mediante un modelo de ejecución aislado, lo que podría reducir la superficie de ataque en aplicaciones distribuidas.

La adopción de WebAssembly en entornos distribuidos no solo podría simplificar el desarrollo y mantenimiento de aplicaciones complejas, sino también mejorar la interoperabilidad entre servicios escritos en diferentes lenguajes de programación. Esto se alinea con la tendencia actual de construir sistemas más flexibles y modulares que puedan adaptarse rápidamente a las necesidades cambiantes del negocio.

La justificación de este estudio radica en la necesidad de encontrar soluciones más eficientes y seguras para el desarrollo de arquitecturas de microservicios en entornos distribuidos. Al analizar y comparar la aplicación de WebAssembly con las tecnologías existentes, este trabajo busca proporcionar una comprensión más clara de sus beneficios y limitaciones, así como su potencial impacto en la industria del software.

\section{Estructura de la memoria}

TBD cuando termine la memoria